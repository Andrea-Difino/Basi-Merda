\documentclass[10pt]{article}
\author{Andrea Difino - Davide Colabove}
\date{A.A. 2024/25 -- Corso di Laurea in Scienze e Tecnologie Informatiche\\SC1167}
\usepackage[T1]{fontenc}
\usepackage[scaled]{uarial}
\renewcommand{\familydefault}{\sfdefault}
\usepackage{tgadventor}
\usepackage{amsfonts}
\usepackage{amssymb}
\usepackage{amsmath}
\usepackage{listings}
\usepackage{array}
\usepackage[margin=2.54cm]{geometry}

\begin{document}
\title{Progetto Basi di Dati - Scuola Guida}
\maketitle

\section{ABSTRACT}{
    Questo progetto si propone di sviluppare un sistema di gestione informatizzato per una scuola guida, partendo dall’analisi dei requisiti e progettando una base di dati capace di gestire utenti, lezioni teoriche e pratiche, veicoli, istruttori, esami e prenotazioni.

    L'allievo è tenermente accolto nei nostri uffici dove si registra l'apertura di un nuovo fascicolo personale e l'ammissione e il salvataggio delle informazioni più importanti per ciascuno di essi, lo aiuterà nel suo percorso guida. Istruttori, identificati con dati anagrafici e patente abilitata, gestiscono lezioni teoriche e pratiche. Lezioni di diverso livello, con veicoli e istruttori assegnati li affiancano negli orari prestabiliti.

    Prenotazione di lezioni, superamento degli esami di teoria e pratica, disponibilità dei veicoli e di istruttori dedicati rientrano nel sistema. Viene anche utile per recalcitranti interni raccogliere le recensioni sui corsi delle differenti sezioni della scuola corsisti di corsi su lezioni o istruttori.

    Il modello relazionale viene tracciato al fine di mantenere integra e coerente l'informazione, applicando il massimo rigore nell'uso di chiavi primarie e vincoli referenziali. Per concludere, il progetto pone i presupposti per lo sviluppo di un sistema informativo per la scuola guida che è attendibile e completo.
}

\section{ANALISI DEI REQUISITI}{
    Il sistema informativo della scuola guida deve supportare le seguenti funzionalità:

    \begin{itemize}
        \item[-] Gestione anagrafica degli iscritti, inclusi dati personali, tipo di patente e data iscrizione.
        \item[-] Gestione dei veicoli, con targa, tipo, modello, stato (disponibile, manutenzione).
        \item[-] Gestione degli istruttori, con abilitazioni, anni di esperienza e categorie coperte.
        \item[-] Pianificazione e gestione delle lezioni (teoriche e pratiche), con aula/veicolo e istruttore.
        \item[-] Sistema di prenotazione per le lezioni da parte degli iscritti.
        \item[-] Gestione esami (teorici e pratici), con risultati, esiti e veicolo utilizzato.
        \item[-] Sistema di feedback e recensioni sugli istruttori e sulle lezioni.
        \item[-] Gestione della disponibilità delle aule e dei veicoli.
        \item[-] Monitoraggio delle presenze alle lezioni e stato dei pagamenti.
    \end{itemize}
}

\newpage

\section{PROGETTAZIONE CONCETTUALE}{
    \subsection{Lista Entità}{
        Le principali entità identificate sono:
        \begin{itemize}
            \item[-] Iscritto (CF, Nome, Cognome, Telefono, Email, Residenza, TipoPatente, DataIscrizione)
            \item[-] Istruttore (Codice, Nome, Cognome, Telefono, Email, Abilitazione, Esperienza)
            \item[-] Veicolo (Targa, Modello, Tipo, Stato, Anno)
            \item[-] Lezione (ID, Tipo, Data, Ora, Aula/Veicolo, Istruttore)
            \item[-] Prenotazione (ID, Data, Stato, Iscritto, Lezione)
            \item[-] Esame (ID, Tipo, Data, Esito, Voto, Veicolo, Iscritto)
            \item[-] Recensione (ID, Oggetto, Iscritto, Data, Gradimento, Commento)
            \item[-] Aula (Codice, Nome, Posti, Attrezzatura)
            \item[-] Pagamento (ID, Iscritto, Importo, Data, Metodo, Stato)
        \end{itemize}
    }
}
\end{document}